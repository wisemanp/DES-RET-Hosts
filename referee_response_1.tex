\documentclass{article}
\usepackage[T1]{fontenc}
\usepackage[utf8]{inputenc}
\usepackage{color}

\hoffset = -1.0in
\voffset = -1.0in
\textwidth = 6.5in
\textheight = 9.0in


\begin{document}
\title{Response to the referee's comments}
\date{}
\maketitle
\section*{Title}
When I first read the title, I wondered whether it was the host galaxies that were detected in the Dark Energy Survey, or whether it was the rapidly evolving transients detected in DES, or both. My understanding after reading the paper is that it is both, and actually the selection of a large number of RETs in DES data is an important aspect of the analysis. I suggest changing the title to make this clear. One example could be “The Host Galaxies of N Rapidly Evolving Transients Discovered by the Dark Energy Survey,” or “The Host Galaxy Properties of N Rapidly Evolving Transients Discovered by the Dark Energy Survey.”

\vskip0.1cm
{\bf RESPONSE: } Thanks, the title has been changed to ``The Host Galaxies of 106 Rapidly Evolving Transients discovered bythe Dark Energy Survey"

\section*{Abstract}
\begin{enumerate}

\item I think the first sentence is misleading, that RETs/FBOTs are “a distinct class of astrophysical event.”
In fact, from the very few events we have studied in detail in the local universe, there is clearly heterogeneity: 18cow shares almost nothing in common with 18gep aside from a fast-evolving and luminous optical light curve. The PS1 objects in Drout+14 span an enormous range in luminosity, and could easily include known classes of SNe, such as Type Ibn and Type II SNe. I would say that RETs/ FBOTs seem to be distinct from “ordinary,” well-understood classes of supernovae, but still have considerable diversity in and among themselves.

\item I also think the second sentence is misleading, that RETs are characterized by light curves that decline much faster than common classes of supernovae and can be seen to redshifts greater than 1. Again, there is considerable heterogeneity in what has been called a RET/FBOT in the literature. In Drout +14, the distinction was that the peak of the light curve could not be powered by nickel decay — but a number of events had decline rates that could easily be explained by nickel decay (see Section 8.1). In SN2018gep, the rise and decline rate would have passed the Drout+14 criteria, but the decline was consistent with a Type Ic-BL SN. 18cow, of course, did decline much more quickly than typical SNe. Regarding the “can be seen to redshifts greater than 1,” this is only true of the very luminous RETs. The sample in Drout+14 spanned an enormous range of peak luminosity. I would remove this sentence, and instead make the point that the main distinguishing feature of an RET is that the *peak* of the light curve is unlikely to be powered by the radioactive decay of Ni-56 — and that the light curve is single peaked, unlike the many double peaked SNe that are known. In general, I think that it is typically the *rise* time that distinguishes RETs, rather than the decline timescale. Perhaps give numbers for the range of right times, decline times, and peak luminosities that are observed for these events?

\vskip0.1cm
{\bf RESPONSE: } These two points have been addressed together, by streamlining the introductory part of the abstract to the following: ``\textit{RETs typically rise to peak in less than 10 days and fade within 30, a timescale unlikely to be compatible with the decay of Nickel-56 that drives conventional SNe. Their peak luminosity spans a range of $-15<M_g<-22.5$, with some events observed at redshifts greater than 1.}"

\item There should be an “of” between “common classes” and “supernovae”
\vskip0.1cm
{\bf RESPONSE: } Changed to ``conventional supernovae"

\item I would state the exact definition of a DES RET somewhere in the abstract
\vskip0.1cm
{\bf RESPONSE: } There is no exact definition of a DES RET that does not take up a full section, as per Section 2.2. Instead we have added the typical rise and decline times.

\item In the abstract, perhaps state what is meant by “metallicity” ([O/H], etc)?
\vskip0.1cm
{\bf RESPONSE: } Added 

\item “We find that RETs prefer galaxies with high specific SFRs” — state the number or range
\vskip0.1cm
{\bf RESPONSE: } Added the mean

\item “appear to show a lack of chemical enrichment” — state the number or range
\vskip0.1cm
{\bf RESPONSE: } Added the mean

\item “no clear relationships between properties of the host galaxies” — what properties?
\vskip0.1cm
{\bf RESPONSE: } Added mass and sSFR 

\end{enumerate}

\section*{Introduction}
\begin{enumerate}

\item “CCSNe, whose lightcurves are primarily powered by” — I would say “are thought to be primarily powered by”
\vskip0.1cm
{\bf RESPONSE: } Changed

\item “On average, SESNe reside in galaxies with higher specific star-formation rates” — than hydrogen- rich SNe, you mean?
\vskip0.1cm
{\bf RESPONSE: } Yes, changed.

\item “More extreme events tend to occur in galaxies low in mass and high in sSFR” — again, please give some numbers for a reader who is not familiar with typical values
\vskip0.1cm
{\bf RESPONSE: } Added values

\item In the paragraph starting “Recently, inspection of high-cadence, large-area survey data” I think you should emphasize that different papers presenting “samples” have used different search criteria from one another. As stated in the abstract, I disagree with the statement that “one...declines rapidly like the PS1 sample,” because a number of the events in the PS1 sample did not decline particularly rapidly. As far as I understand, the PS1 and DES samples did not use the same search criteria.
\vskip0.1cm
{\bf RESPONSE: } Added a sentence stating \textit{although the search and selection criteria vary between the above samples}

\item “There are many diverse explanations for the power source of AT2018cow touted in the literature.” I suggest finding a different word to replace “tout,” such as “proposed.”
\vskip0.1cm
{\bf RESPONSE: } Changed

\item “including: magnetars” — I would instead say engine-driven stellar explosion, which could include magnetars but also accreting black holes. For this hypothesis, please cite Ho et al. 2019 and Margutti et al. 2019, who both proposed that 18cow was an engine-driven stellar explosion based on the X-ray and radio behavior.

\vskip0.1cm
{\bf RESPONSE: } Changed to engine-driven stellar explosions and added the references

\item You describe them later, but I would introduce SN2018gep (Ho+2019) in this section as well, and perhaps also mention the Koala (Ho+2020) and CSS+161010 (Coppejans+2020).
\vskip0.1cm
{\bf RESPONSE: } Added

\item “wind-driven transient” — please cite Piro \& Lu (2020), who made a similar proposal
\vskip0.1cm
{\bf RESPONSE: } Added

\item “It is currently unclear whether these transients represent the local analogues” — delete “the.” Also, I think it would be more useful to state whether these transients truly fit the search criteria used in the PS1 and DES searches. If you placed these transients at the redshift of the DES searches, would they pass the search criteria?
\vskip0.1cm
{\bf RESPONSE: } Have clarified that these objects would fit the criteria out to high redshifts.

\item “For clarity, we will use the term RET to refer only to events in the high-redshift samples of DES and PS1” — this sentence is a bit unclear. You mean that you are applying a luminosity cut? What is the motivation for that?

\vskip0.1cm
{\bf RESPONSE: } Have removed this sentence

\item “before a discussion (Section 6.3) — perhaps you meant to say Section 6?
\vskip0.1cm
{\bf RESPONSE: } Changed, thanks

\item There should be a comma after “i.e.”
\vskip0.1cm
{\bf RESPONSE: } As far as we are aware, in British English i.e. comes without a comma (it seems this way on the MNRAS style guide as well).

\end{enumerate}
\section*{Sample Selection}

\begin{enumerate}
    \item “We derive our sample from the 106 RETs discovered in the 5-year DES-SN transient survey.” sounds a bit strange — perhaps “We derive our sample of host-galaxy properties from,” or “Our sample comprises 106 RETs discovered in”
    
    \vskip0.1cm
{\bf RESPONSE: }
Changed to \textit{Our sample of RETs comprises 106 events discovered in }.

\item I want to make sure the authors are aware of Modjaz+ (2019) which presents an untargeted sample of the kinds of explosions relevant for RETs: Ic-BL SNe and Ic SNe from PTF, and 10 GRB-SNe. Perhaps this could be a useful comparison?
\vskip0.1cm
{\bf RESPONSE: } We have added the SNe Ic-BL as a comparison sample, due to both iPTF16asu and SN2018gep showing Ic-BL-like spectra. 

\item “with remaining eight known as” — should be “with the remaining eight known as”

\vskip0.1cm
{\bf RESPONSE: } Changed

\item “while faster events and low redshift” — you mean “faster events at low redshift”?

\vskip0.1cm
{\bf RESPONSE: } Yes, changed

\item “The key features of RETs that separate them from most traditional SNe types are the fast light curve evolution (rise to peak <15d) and blue colour at peak” — this is not consistent with what you said earlier, which is that the distinguishing feature is the fast *decline* rather than rise
\vskip0.1cm
{\bf RESPONSE: } We see how this was confusing, and have changed this sentence as well as the earlier one to imply that the key feature is the overall fast light curve evolution, rather than the rise or decline in particular

\item When you say “rise to peak” please quantify what this means — time from estimated explosion to peak? time from half-peak to peak in flux space? etc. Otherwise it is difficult to compare to other surveys.

\vskip0.1cm
{\bf RESPONSE: } We now state clearly $t_\mathrm{half}$.

\item “Even though both of these quantities depend on the redshift of the transient” — perhaps give the reader a sense for by how much these quantities can vary over the redshifts considered in the paper

\vskip0.1cm
{\bf RESPONSE: } Added 

\item “the time taken to rise from non-detection to peak” — but doesn’t the magnitude of the non-detection vary with observing conditions, and over time? It seems like it would be difficult for someone using another survey dataset to reproduce your search criteria.

\vskip0.1cm
{\bf RESPONSE: } While this is true, in DES the limiting magnitudes were relatively stable. The relatively slow cadence of 7 days is much more of a limiting factor on constraining the rise time.

\item In this section, it’s not clear to me whether you’re using extinction-corrected colors or not. It seems like that would be important to take into consideration, and to mention in 2.2.2
\vskip0.1cm
{\bf RESPONSE: } Added a sentence clarifying that we are not using extinction-corrected magnitudes for classification, but that the extinction in the DES-SN fields is negligible.

\item Section 2.2.3 is very brief. For a reader who is not familiar with CNNs, perhaps provide a bit of context for this choice, the basic choices you had to make in setting up the model, possible drawbacks, and how the validation was done (a leave-one-out test...?)

\vskip0.1cm
{\bf RESPONSE: } We have added some detail regarding the CNN to this section.

\item “We impose a cut based on a SN classifier” — you mean a SN light-curve classifier?

\vskip0.1cm
{\bf RESPONSE: } Changed

\item  “The decline time to half of the peak brightness” — why not use a criterion like this for the rise time as well? You should be able to set a limit on the rise time based on the non-detection, even if you can’t measure it exactly

\vskip0.1cm
{\bf RESPONSE: } We do - this is the selection based on the red box in Fig. 1.

\item “or showing a longer timescale decline” — but if so, how did it pass the criterion of the decline being < 24 days?

\vskip0.1cm
{\bf RESPONSE: } Some objects (many of which are AGN) decline to half of peak within 24d but are clearly detected for much longer

\item  “for which the redshift has taken on a new value than that presented in P18 due to...more accurate determination” — this seems unnecessarily wordy. I would simply say, “for which further OzDES observations have led to a more accurate redshift than that presented in P18.”

\vskip0.1cm
{\bf RESPONSE: } Changed, thanks

\item[] Table 1: Perhaps add “decimal degrees” to the RA and Dec columns?
\vskip0.1cm
{\bf RESPONSE: } Added `deg' 
\item[] Figure 2: Please indicate the orientation (i.e. a compass showing N/S, E/W). I also suggest putting a scale bar on the figure rather than just stating it in the caption. Consider labeling the relevant boxes with the host redshift as well.

{\bf RESPONSE: } Added all of the above.

\end{enumerate}
\section*{Section 2.3}
\begin{enumerate}
    \item “there is no other large sample of RETs with which to compare host galaxy properties.” Do the properties of the DES RET sample overlap with known classes of supernovae? For example, Type Ibn supernovae, which rise quite quickly to high peak luminosity (M ~ -19.5)? It might be useful to acknowledge that there are probably subclasses represented in the DES sample, which arise from a variety of progenitor channels.
    
    \vskip0.1cm
    {\bf RESPONSE: } Added a sentence to the beginning of 2.3.2: \textit{It is likely that the progenitor and explosion scenario of the DES RET sample is heterogeneous in origin and may include some known classes of transient, and as such we choose samples of host galaxies of various transients from the literature to compare the DES RET hosts to}.
    
\item The reference to SN2018gep here, Ho et al. (2019), is actually to the AT2018cow paper. Please add the correct SN2018gep reference, which is a separate paper, i.e. will end up being Ho et al. 2019a \& Ho et al. 2019b.
\vskip0.1cm
 {\bf RESPONSE: } Changed
 
 \item “with the the” — extra “the”
 \vskip0.1cm
 {\bf RESPONSE: } Removed
 
 \item I might suggest adding iPTF16asu to your list of comparison objects. It was similar to SN2018gep. The reference is Whitesides+ 2017
 \vskip0.1cm
 {\bf RESPONSE: } We have added iPTF16asu to the comparison objects

\end{enumerate}

\section*{Section 2.4} 

\begin{enumerate}
    \item I was a bit surprised by the choice of SDSS as a comparison sample of field galaxies. My understanding is that SDSS galaxies suffer from a number of biases -- they will be highly incomplete for dwarfs and biased towards massive galaxies, and the metallicity measurements are very biased (since they have enough S/N in the critical lines for the measurement, and the strengths of those lines is metallicity-dependent.) Since RETs are likely related to GRBs and SLSNe in terms of their stellar evolutionary pathways, it seems important to have a control sample that is inclusive of low-mass galaxies (see the discussion in Section 4.2 in Modjaz+2019, which presents a sample of Ic and Ic-BL hosts from PTF). Why do the authors not use a volume-complete sample of local galaxies instead or as well, as was done in e.g. Modjaz+2019? Either way, I think the authors need to spend more time motivating their choice of field galaxy sample, and discussing possible drawbacks and biases.
   \vskip0.1cm
 {\bf RESPONSE: } We agree that SDSS is not an ideal comparison sample due to the biases pointed out by the referee. We note, however, that the DES host galaxy photometry is similarly incomplete, albeit down to lower mass/higher redshift than SDSS. Furthermore, the majority of the hosts were followed up with OzDES with a magnitude-based selection and limit. We therefore expect similar biases in the DES RET hosts as seen in SDSS. 
 We acknowledge the use of comparing to LVL and have included that in the figures, but this is mainly for illustrative purposes, since the bulk of low mass LVL galaxies would not be seen in DES-SN at the redshifts of the RET sample.
    
\end{enumerate}

\section*{Section 4.2}
1) “There are several reasons this measurement may differ from one made at the SN explosion site” — could the authors give a quantitative sense for by how much this could differ?
   \vskip0.1cm
 {\bf RESPONSE: } We have added a section describing how these difference are negligible compared to the uncertainty caused by low SNR.

\section*{Section 5}
1) Again, I might suggest including iPTF16asu (Whitesides+2017)
 \vskip0.1cm
 {\bf RESPONSE: }Done
 
 \section*{Section 5.2}
 
 1)  “we compare the chemical state of RET host galaxies with...star-forming field galaxies.” As I stated
above, SDSS metallicities are very biased. I think the authors need to discuss this, or possibly use an additional comparison sample.
\vskip0.1cm
 {\bf RESPONSE: } Done, as discussed previously
 
 \section*{Section 6.2}
 \begin{enumerate}
     \item Ho et al. (2020) (ZTF18abvkwla) and Coppejans (2020) (CSS161010) also performed rate estimates of RETs, so these should be referenced
    \vskip0.1cm
 {\bf RESPONSE: }  Added
 \item As stated in the point above, I think that for a rate estimate it is important to be clear how these different papers define RET...Drout+14 spans a large luminosity distribution, for example
\vskip0.1cm
 {\bf RESPONSE: } Added this caveat
 
 \item “possibly sharing some of the extreme properties of SLSNe or LGRBs” — what kinds of extreme properties?
 \vskip0.1cm
 {\bf RESPONSE: } Added \textit{Such as rapid rotation and low metallicity}.
 
 \item “are an intermediary and/or precursory step” — intermediate and/or precursory step in terms of what parameters? I’m not sure what this means. For example, 18cow is clearly very different from an LGRB-type event — H and He emission lines were observed in the optical spectra, whereas no H or He has ever been observed in a Ic-BL SNe.
 
  \vskip0.1cm
 {\bf RESPONSE: } Clarified to now state: \textit{ his hypothesis can be extended to posit that some RETs represent an intermediate and/or precursory step in the late stages of evolution of a massive star that is close to forming a SLSN or LGRB, whereby the initial collapse of the star occurs leading to shock breakout and subsequent cooling driving the RET light curve(P18), but conditions are not highly tuned enough for a LGRB or SLSN and the respective central engine does not form, hence the lack of longer-term light curves.}
     
    
 \end{enumerate}
 
 \section*{Section 6.3}
 \begin{enumerate}
 \item “have been k-corrected assuming a blackbody SED” — doesn’t there have to be some temperature assumed for this?
 
  \vskip0.1cm
 {\bf RESPONSE: } We fit a blackbody to the data to measure a temperature and radius, which are then used to k-correct the observed magnitudes. We have added a reference to P18 where this procedure is described.
 
 \item Some of the detailed descriptions of individual objects here (18cow, 18gep) seem like they would
belong earlier in the paper, when these objects are mentioned for the first time
\vskip0.1cm
 {\bf RESPONSE: } Moved, thanks
 
 \item “we cannot rule out that the Koala comes from the same population of transients as the DES RETs” Would the light curve of the Koala (and 18gep, and 18kzr) have passed the criteria of the DES RET search, and if so out to what redshift? I think a more quantitative statement here would be useful
 \vskip0.1cm
 {\bf RESPONSE: } Yes, they would have passed out to high ($z\sim1$) redshifts. We have added a comment on this into the introduction
 
\end{enumerate}
\end{document}
